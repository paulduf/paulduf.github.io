\providecommand{\bbobecdfcaptionsinglefunctionssingledim}[1]{
Empirical cumulative distribution of simulated (bootstrapped)
             runtimes, measured in number of objective function evaluations,
             divided by dimension (FEvals/DIM) for the $41$ 
             targets $10^{[-6..2]}$ in dimension #1.
}
\providecommand{\cocoversion}{\hspace{\textwidth}\scriptsize\sffamily{}\color{Gray}Data produced with COCO v5z33}
\providecommand{\numofalgs}{5}
\providecommand{\bbobECDFslegend}[1]{
Bootstrapped empirical cumulative distribution of the number of objective function evaluations divided by dimension (FEvals/DIM) for $41$ targets with target precision in $10^{[-6..2]}$ for all functions and subgroups in #1-D. 
}
\providecommand{\bbobppfigslegend}[1]{
Expected running time (\ERT\ in number of $f$-evaluations
                    as $\log_{10}$ value), divided by dimension for target function value $10^{-6}$
                    versus dimension. Slanted grid lines indicate quadratic scaling with the dimension. Different symbols correspond to different algorithms given in the legend of #1. Light symbols give the maximum number of function evaluations from the longest trial divided by dimension. Black stars indicate a statistically better result compared to all other algorithms with $p<0.01$ and Bonferroni correction number of dimensions (six).  
Legend: 
{\color{NavyBlue}$\circ$}: \algorithmA
, {\color{Magenta}$\diamondsuit$}: \algorithmB
, {\color{Orange}$\star$}: \algorithmC
, {\color{CornflowerBlue}$\triangledown$}: \algorithmD
, {\color{red}$\varhexagon$}: \algorithmE
}
% define some COCO/dvipsnames colors because
% ACM style does not allow to use them directly
\definecolor{NavyBlue}{HTML}{000080}
\definecolor{Magenta}{HTML}{FF00FF}
\definecolor{Orange}{HTML}{FFA500}
\definecolor{CornflowerBlue}{HTML}{6495ED}
\definecolor{YellowGreen}{HTML}{9ACD32}
\definecolor{Gray}{HTML}{BEBEBE}
\definecolor{Yellow}{HTML}{FFFF00}
\definecolor{GreenYellow}{HTML}{ADFF2F}
\definecolor{ForestGreen}{HTML}{228B22}
\definecolor{Lavender}{HTML}{FFC0CB}
\definecolor{SkyBlue}{HTML}{87CEEB}
\definecolor{NavyBlue}{HTML}{000080}
\definecolor{Goldenrod}{HTML}{DDF700}
\definecolor{VioletRed}{HTML}{D02090}
\definecolor{CornflowerBlue}{HTML}{6495ED}
\definecolor{LimeGreen}{HTML}{32CD32}

\providecommand{\ntables}{7}
\providecommand{\pptablesfooter}{
\end{tabularx}
}
\providecommand{\pptablesheader}{
\begin{tabularx}{1.0\textwidth}{@{}c@{}|*{7}{@{}r@{}X@{}}|@{}r@{}@{}l@{}}
$\Delta f_\mathrm{opt}$ & \multicolumn{2}{@{\,}l@{\,}}{1e1} & \multicolumn{2}{@{\,}l@{\,}}{1e0} & \multicolumn{2}{@{\,}l@{\,}}{1e-1} & \multicolumn{2}{@{\,}l@{\,}}{1e-2} & \multicolumn{2}{@{\,}l@{\,}}{1e-3} & \multicolumn{2}{@{\,}l@{\,}}{1e-5} & \multicolumn{2}{@{\,}l@{\,}}{1e-6} & \multicolumn{2}{|@{}l@{}}{\#succ}\\\hline
}
\providecommand{\algEtables}{\StrLeft{trust-constr}{\ntables}}
\providecommand{\algDtables}{\StrLeft{slsqp}{\ntables}}
\providecommand{\algCtables}{\StrLeft{fmincon on bbob-constrained}{\ntables}}
\providecommand{\algBtables}{\StrLeft{cobyla}{\ntables}}
\providecommand{\algAtables}{\StrLeft{alN1}{\ntables}}
\providecommand{\ntables}{7}
\providecommand{\ntables}{7}
\providecommand{\ntables}{7}
\providecommand{\bbobpptablesmanylegend}[1]{%
        Expected runtime (\ERT) to reach given targets, measured
        in number of function evaluations, in #1. For each function, the \ERT\ 
        and, in braces as dispersion measure, the half difference between 10 and 
        90\%-tile of (bootstrapped) runtimes is shown for the different
        target \Df-values as shown in the top row. 
        \#succ is the number of trials that reached the last target
        $\fopt + 10^{-6}$.
        %
        The median number of conducted function evaluations is additionally given in 
        \textit{italics}, if the target in the last column was never reached.
        Entries, succeeded by a star, are statistically significantly better (according to
        the rank-sum test) when compared to all other algorithms of the table, with
        $p = 0.05$ or $p = 10^{-k}$ when the number $k$ following the star is larger
        than 1, with Bonferroni correction by the number of functions (48). A $\downarrow$ indicates the same tested against . Best results are printed in bold.
        \cocoversion}
\providecommand{\algsfolder}{alN1_cobyl_fminc_slsqp_trust/}
\providecommand{\algorithmA}{alN1}
\providecommand{\algorithmB}{cobyla}
\providecommand{\algorithmC}{fmincon on bbob-constrained}
\providecommand{\algorithmD}{slsqp}
\providecommand{\algorithmE}{trust-constr}
\providecommand{\bbobecdfcaptionsinglefunctionssingledim}[1]{
Empirical cumulative distribution of simulated (bootstrapped)
             runtimes, measured in number of objective function evaluations,
             divided by dimension (FEvals/DIM) for the $41$ 
             targets $10^{[-6..2]}$ in dimension #1.
}
\providecommand{\cocoversion}{\hspace{\textwidth}\scriptsize\sffamily{}\color{Gray}Data produced with COCO v5z33}
\providecommand{\numofalgs}{7}
\providecommand{\bbobECDFslegend}[1]{
Bootstrapped empirical cumulative distribution of the number of objective function evaluations divided by dimension (FEvals/DIM) for $41$ targets with target precision in $10^{[-6..2]}$ for all functions and subgroups in #1-D. 
}
\providecommand{\bbobppfigslegend}[1]{
Expected running time (\ERT\ in number of $f$-evaluations
                    as $\log_{10}$ value), divided by dimension for target function value $10^{-6}$
                    versus dimension. Slanted grid lines indicate quadratic scaling with the dimension. Different symbols correspond to different algorithms given in the legend of #1. Light symbols give the maximum number of function evaluations from the longest trial divided by dimension. Black stars indicate a statistically better result compared to all other algorithms with $p<0.01$ and Bonferroni correction number of dimensions (six).  
Legend: 
{\color{NavyBlue}$\circ$}: \algorithmA
, {\color{Magenta}$\diamondsuit$}: \algorithmB
, {\color{Orange}$\star$}: \algorithmC
, {\color{CornflowerBlue}$\triangledown$}: \algorithmD
, {\color{red}$\varhexagon$}: \algorithmE
, {\color{YellowGreen}$\triangle$}: \algorithmF
, {\color{cyan}$\pentagon$}: \algorithmG
}
% define some COCO/dvipsnames colors because
% ACM style does not allow to use them directly
\definecolor{NavyBlue}{HTML}{000080}
\definecolor{Magenta}{HTML}{FF00FF}
\definecolor{Orange}{HTML}{FFA500}
\definecolor{CornflowerBlue}{HTML}{6495ED}
\definecolor{YellowGreen}{HTML}{9ACD32}
\definecolor{Gray}{HTML}{BEBEBE}
\definecolor{Yellow}{HTML}{FFFF00}
\definecolor{GreenYellow}{HTML}{ADFF2F}
\definecolor{ForestGreen}{HTML}{228B22}
\definecolor{Lavender}{HTML}{FFC0CB}
\definecolor{SkyBlue}{HTML}{87CEEB}
\definecolor{NavyBlue}{HTML}{000080}
\definecolor{Goldenrod}{HTML}{DDF700}
\definecolor{VioletRed}{HTML}{D02090}
\definecolor{CornflowerBlue}{HTML}{6495ED}
\definecolor{LimeGreen}{HTML}{32CD32}

\providecommand{\ntables}{7}
\providecommand{\pptablesfooter}{
\end{tabularx}
}
\providecommand{\pptablesheader}{
\begin{tabularx}{1.0\textwidth}{@{}c@{}|*{7}{@{}r@{}X@{}}|@{}r@{}@{}l@{}}
$\Delta f_\mathrm{opt}$ & \multicolumn{2}{@{\,}l@{\,}}{1e1} & \multicolumn{2}{@{\,}l@{\,}}{1e0} & \multicolumn{2}{@{\,}l@{\,}}{1e-1} & \multicolumn{2}{@{\,}l@{\,}}{1e-2} & \multicolumn{2}{@{\,}l@{\,}}{1e-3} & \multicolumn{2}{@{\,}l@{\,}}{1e-5} & \multicolumn{2}{@{\,}l@{\,}}{1e-6} & \multicolumn{2}{|@{}l@{}}{\#succ}\\\hline
}
\providecommand{\algGtables}{\StrLeft{trust-constr}{\ntables}}
\providecommand{\algFtables}{\StrLeft{slsqp}{\ntables}}
\providecommand{\algEtables}{\StrLeft{cobyla}{\ntables}}
\providecommand{\algDtables}{\StrLeft{alO5}{\ntables}}
\providecommand{\algCtables}{\StrLeft{alN5}{\ntables}}
\providecommand{\algBtables}{\StrLeft{alN1}{\ntables}}
\providecommand{\algAtables}{\StrLeft{al01}{\ntables}}
\providecommand{\ntables}{7}
\providecommand{\ntables}{7}
\providecommand{\ntables}{7}
\providecommand{\bbobpptablesmanylegend}[1]{%
        Expected runtime (\ERT) to reach given targets, measured
        in number of function evaluations, in #1. For each function, the \ERT\ 
        and, in braces as dispersion measure, the half difference between 10 and 
        90\%-tile of (bootstrapped) runtimes is shown for the different
        target \Df-values as shown in the top row. 
        \#succ is the number of trials that reached the last target
        $\fopt + 10^{-6}$.
        %
        The median number of conducted function evaluations is additionally given in 
        \textit{italics}, if the target in the last column was never reached.
        Entries, succeeded by a star, are statistically significantly better (according to
        the rank-sum test) when compared to all other algorithms of the table, with
        $p = 0.05$ or $p = 10^{-k}$ when the number $k$ following the star is larger
        than 1, with Bonferroni correction by the number of functions (48). A $\downarrow$ indicates the same tested against . Best results are printed in bold.
        \cocoversion}
\providecommand{\algsfolder}{al01_alN1_alN5_alO5_cobyl_slsqp_trust_et_al/}
\providecommand{\algorithmA}{al01}
\providecommand{\algorithmB}{alN1}
\providecommand{\algorithmC}{alN5}
\providecommand{\algorithmD}{alO5}
\providecommand{\algorithmE}{cobyla}
\providecommand{\algorithmF}{slsqp}
\providecommand{\algorithmG}{trust-constr}
\providecommand{\algorithmA}{alcmaes misc}
\providecommand{\algorithmB}{cobyla}
\providecommand{\algorithmC}{slsqp}
\providecommand{\algorithmD}{trust-constr}
\providecommand{\bbobecdfcaptionsinglefunctionssingledim}[1]{
Empirical cumulative distribution of simulated (bootstrapped)
             runtimes, measured in number of objective function evaluations,
             divided by dimension (FEvals/DIM) for the $41$ 
             targets $10^{[-6..2]}$ in dimension #1.
}
\providecommand{\cocoversion}{\hspace{\textwidth}\scriptsize\sffamily{}\color{Gray}Data produced with COCO v5z33}
\providecommand{\numofalgs}{1}
\providecommand{\algname}{alcmaes misc{}}
\providecommand{\algfolder}{alcmaes_misc/}
\providecommand{\bbobecdfcaptionallgroups}[1]{
Empirical cumulative distribution of simulated (bootstrapped)
             runtimes, measured in number of objective function evaluations,
             divided by dimension (FEvals/DIM) for the $41$ 
             targets $10^{[-6..2]}$ for all function groups and all 
             dimensions. The aggregation over all 48 
             functions is shown in the last plot.
}
\providecommand{\bbobecdfcaptionsinglefcts}[2]{
Empirical cumulative distribution of simulated (bootstrapped) runtimes in number
             of objective function evaluations divided by dimension (FEvals/DIM) for the 
             $41$ targets $10^{[-6..2]}$
             for functions $f_{#1}$ to $f_{#2}$ and all dimensions. 
}
\providecommand{\bbobpptablecaption}[1]{
%
        Expected runtime (\ERT) to reach given targets, measured
        in number of function evaluations in #1. For each function, the \ERT\ 
        and, in braces as dispersion measure, the half difference between 10 and 
        90\%-tile of (bootstrapped) runtimes is shown for the different
        target \Df-values as shown in the top row. 
        \#succ is the number of trials that reached the last target 
        $\fopt + 10^{-6}$.
        The median number of conducted function evaluations is additionally given in 
        \textit{italics}, if the target in the last column was never reached. 
        
}
\providecommand{\bbobppfigdimlegend}[1]{
%
        Scaling of runtime with dimension to reach certain target values \Df.
        Lines: expected runtime (\ERT);
        Cross (+): median runtime of successful runs to reach the most difficult
        target that was reached at least once (but not always);
        Cross ({\color{red}$\times$}): maximum number of
        $f$-evaluations in any trial. Notched boxes: interquartile range with median of simulated runs; 
        All values are divided by dimension and  
        plotted as $\log_{10}$ values versus dimension. %
        %
        Shown is the \ERT\ for fixed values of $\Df = 10^k$ with $k$ given
        in the legend.
        Numbers above \ERT-symbols (if appearing) indicate the number of trials
        reaching the respective target.  Horizontal lines mean linear scaling, slanted
        grid lines depict quadratic scaling.
}
\providecommand{\bbobpprldistrlegend}[1]{
%
         Empirical cumulative distribution functions (ECDF), plotting the fraction of
         trials with an outcome not larger than the respective value on the $x$-axis.
         #1%
         Left subplots: ECDF of the number of function evaluations ((f+g)-evals) divided by search space dimension $D$,
         to fall below $\fopt+\Df$ with $\Df=10^{k}$, where $k$ is the first value in the legend.
         The thick red line represents the most difficult target value $\fopt+ 10^{-6}$. %
         Legends indicate for each target the number of functions that were solved in at
         least one trial within the displayed budget.
         Right subplots: ECDF of the best achieved $\Df$
         for running times of $0.5D, 1.2D, 3D, 10D, 100D, 1000D,\dots$
         function evaluations
         (from right to left cycling cyan-magenta-black\dots) and final $\Df$-value (red),
         where \Df and \textsf{Df} denote the difference to the optimal function value. 
         Shown are aggregations over problems where the objective
            functions are in the same BBOB function class and the aggregation
            over all 48 functions in the last row.
}
\providecommand{\bbobloglossfigurecaption}[1]{
%
        \ERT\ loss ratios (see Figure~\ref{tab:ERTloss} for details).

        Each cross ({\color{blue}$+$}) represents a single function, the line
        is the geometric mean.
        
}
\providecommand{\bbobloglosstablecaption}[1]{
%
        \ERT\ loss ratio versus the budget in number of $f$-evaluations
        divided by dimension.
        For each given budget \FEvals, the target value \ftarget\ is computed
        as the best target $f$-value reached within the
        budget by the given algorithm.
        Shown is then the \ERT\ to reach \ftarget\ for the given algorithm
        or the budget, if 
        reached a better target within the budget,
        divided by the \ERT\ of  to reach \ftarget.
        Line: geometric mean. Box-Whisker error bar: 25-75\%-ile with median
        (box), 10-90\%-ile (caps), and minimum and maximum \ERT\ loss ratio
        (points). The vertical line gives the maximal number of function evaluations
        in a single trial in this function subset. See also
        Figure~\ref{fig:ERTlogloss} for results on each function subgroup.\cocoversion
        
}
\providecommand{\pptablefooter}{
\end{tabular}
}
\providecommand{\pptableheader}{
\begin{tabular}{@{}c@{}|*{7}{@{}r@{}@{}l@{}}|@{}r@{}@{}l@{}}
$\Delta f$ & \multicolumn{2}{c}{1e+1} & \multicolumn{2}{c}{1e+0} & \multicolumn{2}{c}{1e-1} & \multicolumn{2}{c}{1e-2} & \multicolumn{2}{c}{1e-3} & \multicolumn{2}{c}{1e-5} & \multicolumn{2}{c}{1e-6} & \multicolumn{2}{|@{}r@{}}{\#succ}\\\hline
}
\providecommand{\bbobecdfcaptionsinglefunctionssingledim}[1]{
Empirical cumulative distribution of simulated (bootstrapped)
             runtimes, measured in number of objective function evaluations,
             divided by dimension (FEvals/DIM) for the $41$ 
             targets $10^{[-6..2]}$ in dimension #1.
}
